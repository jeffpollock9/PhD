
\singlespacing

\chapter{Conclusion}
\label{ch:Conclusion}

\onehalfspacing

\section{Results and discussion}

At the beginning of this thesis, we described four main aims regarding the development of modelling approaches in
association football. To reiterate in short, they are:
\begin{enumerate}
\item The creation of models which are widely applicable, have improved predictive power, and are more parsimonious than
models currently in literature
\item Inference of models in a Bayesian framework so that prior knowledge can be exploited and parameter uncertainty
accounted for
\item Methods of inference which are quick to compute
\item The creation of models which can capture behavioural aspects of teams
\end{enumerate}

Throughout this thesis we have shown examples of numerous Bayesian methods of inference, for example \gls{MH}, particle
filtering methods, and \gls{RJMCMC}. We have suggested benefit in these methods which allow the addition of extra
information into the model inference process via prior distributions, perform parameter updating very quickly, and also
naturally inform model choice/model averaging decisions. Not only this but the Bayesian framework easily accounts for
uncertainty in parameter estimation through the posterior distribution. We strongly feel that the Bayesian framework
should be strongly considered when performing inference of sports related models - and hope that this thesis has shown
some of the many benefits.

In Chapter \ref{ch:An_adaptive_behaviour_model_for_association_football_using_rankings_as_prior_information} of the
thesis we presented a novel non-homogeneous Poisson process model which we then presented several modifications to in
Chapters \ref{ch:Fast_updating_of_dynamic_and_static_parameters_using_particle_filters} and
\ref{ch:A_Utility_Based_Model} (some more potential modifications are discussed below). The model is more parsimonious
than similar models in literature by defining each team's ability by a single team-specific parameter (which we called
the `resource'), as opposed to a team-specific attack and defense parameter (\cite{Maher1982, DixonColes1997,
DixonRobinson1998}). This simpler model was shown to outperform other models in the literature based on a number of
tests, largely concerning one-week-ahead forecasting ability. Furthermore, since the model can be used to simulate match
goal times, it can be used to make predictions around any event which is defined by goal times - of which the vast
majority of betting markets comprise.

We displayed how particle filtering methods could be employed to update posterior distributions extremely quickly and
efficiently, and also how they could cope with a mixture of static and dynamic model parameters. Similarly to
\cite{Owen2011}, we proposed a model whereby the team resource parameters were allowed to vary dynamically throughout a
season, in contrast however, we found no compelling evidence that this addition to the model was beneficial for our
(different to that of \cite{Owen2011}) test data. The superior computational speed of particle filtering methods however
allows users to practically update posterior distributions while matches are being played. Clearly useful if one wished
to have the most up-to-date parameter estimates during matches for betting or other purposes.

Finally, in terms of the results of the thesis, we presented a final extension to the model which included modifying the
team resource allocation parameter (\(\alpha_k(t)\)) in order to account for current league position and obtainable
league positions. This is indeed a new and novel behavioural aspect which previous models in literature have not been
able to fully capture. Model inference then suggested that teams may indeed play in order to avoid ending the season in
position 5 (which grants qualification to the Europa League) by noting a local minimum in the estimated utility function
at this position. That is, there is some evidence to suggest that controversially, teams in the \gls{EPL} would rather
finish the season in position 6 than 5. Furthermore, we showed (via an example of a a match between Manchester City and
Queens Park Rangers in the final week of the season) how the model captures the change in behaviour of teams as they
react to news from other concurrent matches which affects their league situation. The change in behaviour was shown to
have a large real effect on the rates of scoring, in one example increasing the scoring rate by almost 50\%, and thus
has clear implications for certain betting markets.

% Statistical models for sporting outcomes are becoming increasingly in demand with the increase in data collected for
% sports. We naturally consider one of the most popular sports, association football, but similar models may be applicable
% for other sports, for example hockey. The models discussed in this thesis are of most use for betting purposes, and are
% largely interpretable by modelling the behaviour of teams.
% 
% We have shown that the model in it's most basic form is capable of of producing a profit when betting against a large UK
% bookmaker, Bet365, and furthermore discussed methods in which model inference (which can inform betting strategies)
% could be practically undertaken between and during matches. The model also outperformed several other models, including
% the model of \cite{DixonRobinson1998}, when tested on one-week-ahead predictive performance.

Statistical models all make assumptions, either explicit or implicit, and indeed ours is no exception to this. We note
some of our modelling assumptions here:
\begin{enumerate}
  \item Red cards have no effect
  \item The perception of utility is common across all teams
  \item With regards to the match situation (and not the league situation), teams only consider whether they are
  winning, drawing, or losing
  \item With regards to the league, teams consider positions which can be reached following the scoring of at
  most two more goals for either team
\end{enumerate}

The first point portrays the main limitation of the model, that it should not be used for prediction during in-play
matches for which a player has been dismissed via a red card, and that the model likelihood does not explicitly account
for matches which have contained red cards. Whilst relatively rare events, and typically occurring at later times in
matches, red cards \textit{surely} effect the scoring rate during matches. Somewhat surprisingly however,
\cite{volf2009} did consider red cards as a covariate in a Poisson process model for goal times and found it to be
non-significant (a confidence interval for the coefficient contained the value 0).

Having the scoring rates of the competing teams depend on red cards would add another dimension of complexity to the
model. Simulation of matches would require not only the simulation of goal times, but also cards. With our aim to create
parsimonious models, for which inference and prediction was computationally efficient, we chose to not add red card
information - in line with many authors in the literature concerning the modelling of association football outcomes
(for example \cite{DixonColes1997, DixonRobinson1998, mchale2011}).

The remaining three points concern our formulation of the team resource allocation parameter (\(\alpha_k(t)\)) for which
we discuss potential modifications for below. With one of the aims of this thesis to create parsimonious models which
are still capable of capturing a large variety of effects, we always opted for the simplest sensible formulation
possible. We thus assumed that all teams react the same way to league or match position, as opposed to specifying up to
20 team-specific behavioural parameters. At first we also only consider whether a team was losing, drawing, or winning
during a match, which is intuitive and also largely agrees with the modelling of \cite{DixonRobinson1998} who
additionally considered different formulations for when the match score was 0-1 or 1-0. However with the addition of the
league utility considerations into the model, the model considers all score lines attainable through the scoring of up
to two more goals for either team, as well as the winning, drawing, or losing state.

\section{Future work}

Evolution of statistical models in sport often comes as experts in the sport suggest effects which statistical models
should be able to capture. In this thesis, we, for example, hypothesised that teams should change their behaviour based
on the league situation, and were able to specify a model formulation which could capture said effects. There are many
other effects which an expert would suggest a statistical model for association football should be able to capture. This
is both a blessing and a curse. On one hand there are always ideas for how to specify your next model, on the other,
whatever model you can think of will have something that it doesn't account for or is misspecified or misrepresented.
This ties in quite nicely with a well-known quote from \cite{box1987}, `all models are wrong, but some are useful'. This
is why research on statistical models for sport will evolve as long as sport itself does.

To extend the utility based model presented in Chapter \ref{ch:A_Utility_Based_Model} one might look at ways of adding
an idea of `match importance' (\cite{scarf2008}) into the function \(\beta(w)\) so that teams play in accordance to
their league position when the match is considered important with regards to the league. Unfortunately, the raw ideas of
\cite{scarf2008} would not be usable, some other form of `match importance' would need to be formulated.

More generally, there are many factors which the function resource allocation function \(\alpha_k(.)\) could account
for, for example any linear or non-linear interaction of in-play data such as number of corners kicks, number of
throw-ins, number of shots, number of passes, or number of yellow/red cards. One might also expect that different teams
react differently depending on the match state, and also whether or not they are the home or away team. For example we
would expect a strong team at home to react differently to weak team away when they are in the losing state.
Many sporting experts will have (likely differing) opinions on how these data imply changes to the rates of scoring/the
team behaviour in a match - and as these data become more readily available it is likely that statistical models will
appear in literature which account for them all.

More complex models may also require more complex \gls{MCMC} algorithms in order to perform inference efficiently.
Hamiltonian \gls{MCMC} (see for example \cite{neal2011}) is one such algorithm, and it is becoming more popular due to
recent developments in the program Stan (\cite{stan}) which implements Hamiltonian \gls{MCMC} while maintaining a very
similar user interface to WinBUGS. In addition, model comparison will always be a difficult topic, and information
criteria other than \gls{DIC} could also be considered, for example the \gls{WAIC} (see \cite{vehtari2015}).

Somewhat unfortunately, historical in-play betting odds data are not currently easily accessible. This is where one
might expect betting strategies informed by statistical models to be most profitable, due to the continuously changing
odds as the match unfolds and particular events (namely goals) occur. In particular we would expect the model to be
profitable if the bookmaker odds were manually driven. Our non-homogeneous Poisson process model would naturally be able
to provide predicted probabilities for events of matches as they are in-play, although one would need to be careful in
the cases where matches contain `red-card' events, as mentioned above. The first research to appear in the literature
showing the potential profits of in-play, statistical model informed, betting strategies could be truly ground breaking,
and could also be worrying news for bookmakers.

Finally, one might also consider the use of a similar non-homogeneous Poisson process model for the occurrences of
points in other sports. For example Rugby Union where teams divide their resource between attempting to score a try or a
penalty kick/drop goal. In a similar fashion, American football teams may divide their resource between attempting to
score field goals or touchdowns, which suggests an potential extension to the point-process model of \cite{baker2013}.
Ice Hockey is another popular sport where such a model could be effective - in particular since Ice Hockey teams are
known to behave particularly offensively when losing in the final minutes of a match, and often deploy their goalkeeper
as an extra attacking player in an attempt to score (this tactic however most often leads to conceding a goal).



