
\singlespacing

\chapter{Introduction}
\label{ch:Introduction}

\onehalfspacing

\section{An overview of gambling in sport}
\label{sec:An_overview_of_gambling_in_sport}

Gambling has been a hugely popular activity throughout human life, and has recorded mentions in ancient Roman and Greek
history. While the practice of gambling has been frowned upon or even illegal in certain periods of time or locations
over the globe, in recent times in the UK gambling (under regulation) has been widely accepted by the public. Most UK
high streets now contain at least one bookmaking office (shop) and even more gambling is done online. Horse racing and
association football (commonly referred to as \textit{soccer} in the United States and football elsewhere) are the most
gambled upon sports in the UK, with sports like tennis and greyhound racing also being popular.

After a government go-ahead in 1960, UK bookmaking offices began to open in 1961 with the aim of ending unregulated,
illegal gambling. Before then, bets could only be legally placed at racing tracks. The number of bookmaking offices
quickly grew, and in 2013 was estimated at around 8,700 (\cite{bettingShops}). It is however thought that the number of
bookmaking offices is declining due to the popularity of on-line gambling.

Founded in 2000, Betfair (\url{https://www.betfair.com/exchange}) offered the first on-line sports betting exchange. It
allowed bettors to act either as a traditional bettor, `backing' an event, or act in the traditional role of the
bookmaker, `laying' an event. This allowed bettors to bet against each other and thus offer better odds than a
bookmaker. Bookmakers typically worked an expected profit of around 15\% into their odds, whereas Betfair would only
take a small percentage (around 5\%) of any winnings. An example of the Betfair exchange can be seen in Figure
\ref{fig:betfairOdds}, which shows the market of a race at the hugely popular Royal Ascot.
\begin{figure}[htp]
\begin{center}
  \includegraphics[width = 14cm]{betfairOdds2}
  \caption{\label{fig:betfairOdds} A screenshot of the Betfair exchange showing available back and lay odds for the
  runners of the 14:30 at Royal Ascot}
\end{center}
\end{figure}
The Figure shows the odds which are available to back (in bold under `Back all') and the amount of money which is in the
exchange at those odds (directly underneath the odds). Similarly, the odds which you can lay are shown. The screenshot
was taken about an hour before the start of the race, and already the exchange had matched bets up to a total value of
\pounds575,652. It is not uncommon to see several millions matched on big horse racing or association football markets.

Somewhat in response to Betfair, high street bookmakers had to invest in websites which easily allowed on-line gambling
(as we will see in the forthcoming section) and also operated at a lower expected profit, in order to offer competitive
odds. This led to a massive increase in turnover for bookmakers as many people are not comfortable with the Betfair
exchange and prefer a simpler betting approach with a bookmaker. This also suggests an additional opportunity for
bettors, betting against high street bookmakers who now offer better odds.

\section{Gambling in association football}
\label{sec:Gambling_in_association_football}

The first popular association football bet in the UK was known as the `football pools'. Littlewoods football pools was
the first of its kind, beginning in 1923 when the football pools coupons were offered outside Manchester United's ground
Old Trafford. The bet quickly spread across the whole of the UK, probably because, for a small stake the bet offered the
chance of a share of a massive jackpot (pool of money). The aim of the bet was to select the outcome of several matches
which at the time, were all played concurrently at 3pm on a Saturday. More recently, since association matches are
televised, matches occur at different times during the week, although the bulk are played in the traditional 3pm
Saturday slot. At its peak of popularity, it is estimated around that the pools had around 10 million players in the UK,
this figure however severely plummeted following the introduction of the UK national lottery in 1994, which offered even
bigger jackpots. It is also likely that bettors now favour association football bets with bookmakers, who can offer odds on a
much larger selection of events.

Bookmakers now offer a large number of association football betting markets to customers. This is no real surprise since
to quote \cite{constantinou2012}, `[association football] is the world's most popular sport \ldots\ and constitutes the
fastest growing gambling market'. Some of the most popular betting markets are: `1X2' (also called match betting), in
which the bettor chooses the final outcome of the match (home team win, draw, or away team win), `total goals
under/over', in which the bettor chooses if the total number of match goals will be under or over a certain line
(usually 2.5), and `correct score', in which the bettor chooses the exact final score in the form x-y. There are also
more obscure betting opportunities, for example betting on the number of corners in the last 15 minutes of a match. In
fact, as can be seen in Figure \ref{fig:williamHillOdds}, bookmakers such as William Hill
(\url{http://sports.williamhill.com/bet/en-gb}) may offer in hundreds of betting markets (218 for this particular
example) on an association football match.
\begin{figure}[htp]
\begin{center}
  \includegraphics[width = 14cm]{williamHillOdds}
  \caption{\label{fig:williamHillOdds} A screenshot from the website of one of the top UK bookmakers William Hill,
  showing the odds for some of the markets available on an international match between Peru and Venezuela}
\end{center}
\end{figure}

There has also been a recent increase in what is known as `in-play' betting, where bets can be placed during a match.
Previously, as soon as a match had started, the betting markets would suspend, however with in-play betting this is no
longer the case. In-play betting markets see the odds change continuously throughout time in a match, and react to
events such as goals or player dismissals.

\section{Bookmaking}
\label{sec:Bookmaking}

Bookmakers typically estimate the probability of an event (for example a score of 3-0 or a home team win), and then add
what is called `over-round' in order to ensure an expected net profit from their published odds - effectively worsening the
fair odds. Odds may be offered in various forms (for example fractional or American) but we consider the odds in decimal
form, which are the most natural to work with mathematically. Decimal odds are given by:
\begin{align}
d_{m, E} = \frac{1}{p_{m, E} + o_{m, E}}
\end{align}
where \(d_{m, E}\) is the bookmaker's decimal odds, \(p_{m, E}\) is the bookmaker's estimated probability, and \(o_{m,
E}\) is the added over-round, for event \(E\) in match \(m\). From a bettor's point of view, a 1-unit bet at decimal
odds \(d\) results in either a loss of 1 unit, or a profit of \(d-1\) units.

In the past, bookmakers would devise odds based on their expert knowledge of sport and any bets they had already taken.
For example they might begin with their best guess of the statistically profitable odds (which include over-round), then
as bets are placed, shorten (decrease) the odds on events which have been heavily bet (discouraging future bets) and
lengthen (increase) the odds on events which have not been bet so much (encouraging future bets). This strategy aims to
keep a `balanced book', that is, a low-risk spread of bets taken over all events which ideally results in arbitrage for
the bookmaker, and largely no cases where the bookmaker has a large sum to pay should a particular event occur. The
strategy is in a sense Bayesian in flavour since the Bookmaker first assigns some prior odds based on expert knowledge,
and then updates the odds based on new information (bets) becoming available.

We note two points from the above bookmaking strategy, firstly, it is very labour intensive, requiring expert knowledge
in both the prior setting of the odds and any movement of the odds as bets are placed. Secondly, the bookmaker moves
odds based on his current book of bets and not necessarily on the true underlying probability of an event. Therefore,
should a bookmaker take a lot of bets on a particular event, for example a UK bookmaker taking bets on England to win
the association football World Cup, then the odds on that particular event may be particularly bad value and there may
be scope for good-value, long-term profitable betting strategies focused on less fancied teams within the betting market.

The first point mentioned above has been a real problem for bookmakers, and has been remedied in recent years by the use
of statistical models which can calculate odds for hundreds of betting market events. They are most prevalent for
in-play betting markets, since theoretically the odds should change continuously throughout time and a bookmaker who was
manually trading may not be able to keep up with the ever changing market. Furthermore, if the bookmaker believes they
have a good statistical model, they may choose not to change the odds based on incoming bets, instead opting for a
high-risk, high-rate-of-return strategy.

The second point mentioned above is mainly of interest to bettors, and suggests the use of statistical models to bet
against the bookmaker, of which we present an example in Chapter
\ref{ch:An_adaptive_behaviour_model_for_association_football_using_rankings_as_prior_information} Section
\ref{sec:Use_of_models_to_inform_betting_strategies}.

In practice, it is likely that different bookmakers opt for different strategies with regards to the use of statistical
models and movement of odds based on bets. However, statistical models have considerably reduced the amount of manual
labour required for a bookmaker to offer a large number of betting markets - as we will discuss further in Section
\ref{sec:Statistical_models_in_association_football}.

% However, much of the betting markets which bookmakers now offer are only
% possible due to the use of statistical models

\section{The English Premier League}
\label{sec:The_English_Premier_League}
 
The \gls{EPL} is one of the top association football leagues and attracts interest internationally. At the end of the
2011/2012 season over \pounds1.1bn was paid out to the 20 participating teams in broadcast money alone
(\url{www.premierleague.com}). We thus choose this league as a test system for our subsequent statistical models.

Each \gls{EPL} season involves 20 teams playing against each other twice, once at home and once away, giving a total of
380 matches which are played over 38 weeks (10 matches per week). Teams are awarded three points for a win, one point
for a draw, and zero points for a loss, goals scored, conceded, and difference are also recorded for calculation of the
league standings. Teams are first ranked on points, then goal difference, and then goals scored. Ending the season in
positions 1-4 grants qualification to what is known as the Champions League, a short tournament which contains the top
teams in Europe, similarly, position 5 grants qualification to the Europa League. At the other end of the table, teams
in positions 18-20 are relegated to the league below (known as the Championship), while the top 3 teams from the
Championship are promoted into the \gls{EPL}. The league table at the end of season 2011/2012 is shown in Table
\ref{tab:league} where `p' denotes the league position, `pld' the number of matches played, `w' the number of matches
won, `d' the number of matches drawn, `l' the number of matches lost, `gf' the number of goals for (scored), `ga' the
number of goals against (conceded), `gd' the goal difference (`gf' - `ga'), and `pts' the league points.
\begin{table}
\centering
\fbox{
\begin{tabular}{clcccccccc}
p  & team                         & pld& w  & d & l  & gf & ga & gd      & pts\\
\hline
1 	& Manchester City   	      & 38 & 28	&5 	& 5  & 93 &	29 & \(+64\) & 89 \\ 	
\hline
2 	& Manchester United 	      & 38 & 28	&5 	& 5  & 89 &	33 & \(+56\) & 89 \\
3 	& Arsenal 	                  & 38 & 21	&7 	& 10 & 74 &	49 & \(+25\) & 70 \\
4 	& Tottenham Hotspur 	      & 38 & 20	&9 	& 9  & 66 &	41 & \(+25\) & 69 \\ 	
\hline
5 	& Newcastle United 	          & 38 & 19	&8 	& 11 & 56 &	51 & \(+5\)  & 65 \\ 	
\hline
6 	& Chelsea 	                  & 38 & 18	&10 & 10 & 65 &	46 & \(+19\) & 64 \\ 	
7 	& Everton 	                  & 38 & 15	&11 & 12 & 50 &	40 & \(+10\) & 56 \\
8 	& Liverpool 	              & 38 & 14	&10 & 14 & 47 &	40 & \(+7\)  & 52 \\ 	
9 	& Fulham                      &	38 & 14	&10 & 14 & 48 &	51 & \(-3\)  & 52 \\
10 	& West Bromwich Albion 	      & 38 & 13	&8 	& 17 & 45 &	52 & \(-7\)  & 47 \\
11 	& Swansea City                &	38 & 12	&11 & 15 & 44 &	51 & \(-7\)  & 47 \\
12 	& Norwich City 	              & 38 & 12	&11 & 15 & 52 &	66 & \(-14\) & 47 \\
13 	& Sunderland 	              & 38 & 11	&12 & 15 & 45 &	46 & \(-1\)  & 45 \\
14 	& Stoke City                  &	38 & 11	&12 & 15 & 36 &	53 & \(-17\) & 45 \\
15 	& Wigan Athletic              &	38 & 11	&10 & 17 & 42 &	62 & \(-20\) & 43 \\
16 	& Aston Villa 	              & 38 & 7 	&17 & 14 & 37 &	53 & \(-16\) & 38 \\
17 	& Queens Park Rangers 	      & 38 & 10	&7 	& 21 & 43 &	66 & \(-23\) & 37 \\
\hline
18 	& Bolton Wanderers  	      & 38 & 10 &6 	& 22 & 46 &	77 & \(31\)  & 36 \\ 
19 	& Blackburn Rovers  	      & 38 & 8  &7 	& 23 & 48 &	78 & \(-30\) & 31 \\
20 	& Wolverhampton Wanderers     & 38 & 5  &10 & 23 & 40 &	82 & \(-42\) & 25 \\
\end{tabular}}
\caption{\label{tab:league} The league table at the end of the EPL 2011/2012 season}
\end{table}
At the end of the 2010/2011 season (the previous season) Norwich City, Swansea City, and Queens Park Rangers were
promoted into the \gls{EPL}, replacing Birmingham City, Blackpool, and West Ham United.

Each match in the \gls{EPL} consists of two 45-minute halves. However, at the end of each half the referee may allow
the match to continue for longer than the allocated 45 minutes in what is known as `injury time'. Injury time typically
adds around 2 minutes to the first half, and 3.5 minutes to the second half.

As touched upon in Section \ref{sec:Gambling_in_association_football}, each week the 10 \gls{EPL} matches were
traditionally all played at the same time of 3pm on a Saturday. This is however no longer the case, with television
companies demanding that matches be played at different times so more can be televised. The tradition however holds on
the last set of 10 matches in week 38, which often sees teams concurrently playing to avoid league relegation, or at the
other end of the league table, for league victory. Furthermore, as information now flows quickly between matches, at
all times teams are aware of the state of other concurrent games which may affect them.

\section{Statistical models in association football}
\label{sec:Statistical_models_in_association_football}

Coinciding with the rise in on-line and in-play betting as been a demand for statistical models which can capture the
details of sporting contests. As seen in Section \ref{sec:Bookmaking}, bookmakers offer hundreds of markets and thus
require a statistical model which is capable of predicting the probability of each event in each market, both before and
during a match. Statistical models may also be used by bettors, to inform their own betting strategy against bookmakers,
or other bettors on an exchange like Betfair.

Much of the published work on this area typically extends a model proposed by \cite{Maher1982} who suggested modelling
the numbers of goals scored by the home and away teams goals as independent Poisson random variables, where the mean of
each distribution depends on the strength of the home and away team's attacking and defensive capabilities.
\cite{DixonColes1997} proposed an adjustment to the probability of certain match scores in the independent Poisson model
and also suggested that a model's forecasting ability improved if the likelihood incorporated weightings that decreased
exponentially with the time elapsed since the observation of an event. \cite{KarlisNtzoufras2003} built on these ideas
with a diagonally inflated bivariate Poisson model for modelling scores. \cite{mchale2011} also presented a method of
modelling the number of home and away team goals, by assuming the number of goals each team scored had a marginal
negative binomial distribution, and considered the joint distribution of home and away team goals using copula methods
(see \cite{nelsen2007}).

\cite{DixonRobinson1998} introduced a richer model which modelled match goal times as opposed to only the final score.
They demonstrated that when goal times were modelled as a non-homogeneous Poisson process, the rate of scoring of teams
changed depending on the current score. \cite{volf2009} later presented a similar (albeit much less cited) model which
he applied to international matches. While not applied to modelling association football goal times, \cite{baker2013}
developed a 10-dimensional point-process model for the different methods of scoring in American football, and is also an
excellent reference for the type of models mentioned here. 

\cite{Owen2011} presented a dynamic generalized linear model which extended the model of \cite{Maher1982} to the dynamic
spectrum where each team's attack and defense parameter followed a random walk throughout the football season.
It was shown that, when applied to Scottish Premier League data, it was beneficial in terms of predictive performance to
allow a dynamic model component. \cite{koopman2015} then modelled each team's attack and defense parameter as an
auto-regressive process which fed into a bivariate Poisson model for the number of home and away team goals. They used a
mixture of Monte Carlo and maximum likelihood methods for efficient estimation of all model parameters, and showed some
evidence that the model produced a significant profit when compared to bookmaker's odds.

All of the above mentioned models describe the strength of each team with two parameters, one which represents the
team's attacking capability, and one which represents their defensive capability. The \gls{EPL} involves 20 teams
competing throughout a season and thus these models use 39 parameters to describe the team strengths (one parameter must
be constrained for model identifiability). Furthermore, ranking of the teams based on these parameters is not
straightforward.

A further class of models is only concerned with the final match outcome (home team win, draw, or away team win).
\cite{FahrmeirTutz1994, KnorrHeld2000, CattelanVarinFirth2013} modelled this outcome using Bradley-Terry type models
where parameters representing a team's strength could vary dynamically with time. However these models have the
disadvantage of not being able to process all the information contained in a match's score, as for example, a score of
5-0 is treated the same as a score of 3-2.

\cite{scarf2008} also employed a Bradley-Terry type model, which they used in order to estimate an idea of `match
importance' by considering the difference in probability of events like winning the league, when a team won or lost a
match. These ideas fit in with the concept that teams may change their behaviour if a match is deemed important.
Somewhat similarly, the Bradley-Terry type model of \cite{goddard2003} used covariates related to if matches had league
promotion or relegation implications, in a thought that teams may try harder and thus have greater chance of winning in
these matches.

From the view of a bookmaker, Bradley-Terry type models are limited in that they are only able to predict probabilities
(and thus provide odds) for the 1x2 market before a match begins. The Poisson models which extend the work of
\cite{Maher1982} are a little more useful, in that they can predict probabilities for any correct score x-y and thus are
able to provide odds for markets such as total goals under/over, but again, only before a match. The non-homogeneous
Poisson process model of \cite{DixonRobinson1998} however models goal times, and thus is able to provide odds for the
most markets both before the match and in-play. This type of model is thus the most useful to bookmakers who wish to use
models for the purposes of making a betting market, and bettors who wish to bet both before a match and in-play on a
variety of betting markets.

It should also be mentioned that forecasting match outcomes is not the only use of statistical models in association
football. They have become increasingly popular in advising managers on match tactics and player transfers. Data is now
collected for each intricate movement a player makes, from missing a tackle to scoring a goal from 30 metres out, and
teams are becoming aware of the benefits of analysis of this data. Many teams now house data analysis departments whose
job is to value players in order to advise which player transfers represent good value, and to provide feedback on each
individual player's performance. Statistics of this kind have been commonly used in other sports (particularly Baseball,
which even has a book and film about this very topic, `Moneyball', \cite{moneyball}) but have only recently become
utilised in association football. In fact, in 2012, one of the top \gls{EPL} teams, Manchester City, made player data
publicly available in order to promote analysis of player performance.
 
\section{Data}
\label{sec:Data}

We use data that record for each match the particular minute(s) during which a goal is scored and the team scoring the
goal. Our analyses in Chapter
\ref{ch:An_adaptive_behaviour_model_for_association_football_using_rankings_as_prior_information} and Chapter
\ref{ch:Fast_updating_of_dynamic_and_static_parameters_using_particle_filters} concern all matches in the 2011/2012
\gls{EPL} season, with specification of prior distributions based on the previous season (2010/2011). While analyses in
Chapter \ref{ch:A_Utility_Based_Model} concern all matches from five seasons of \gls{EPL} data, from seasons 2007/2008
to 2011/2012. The goal-time data are available at \url{www.scoresway.com}.

As mentioned in Section \ref{sec:The_English_Premier_League} at the end of each 45 minute half, the referee allows extra
injury time to be played. Goals scored in first half injury time are simply recorded in the data as having been scored
at minute 45, and similarly for goals in second half injury time at minute 90. Hence, the histogram shown in Figure
\ref{goalTimes} displays visible spikes at these times. We also note here that the vast majority of the figures in this
thesis have been created via the use of R and ggplot2 (\cite{R, ggplot2}).
\begin{figure}
\begin{center}
\includegraphics[width = 14cm]{goal-times}
\caption{A bar plot of goal times (in minutes) for the EPL season 2011/2012}
\label{goalTimes}
\end{center} 
\end{figure}

We also use historical data on bookmaker's odds from the 2011/2012 \gls{EPL} season which are available as a direct
download on the website \url{www.football-data.co.uk}. These data contain pre-match odds from the UK bookmaker Bet365
for the 1X2 betting market. In the notation of Section \ref{sec:Bookmaking}, the data comprise \(d_{m, H}\), \(d_{m,
D}\), and \(d_{m, A}\) for matches \(m = 1, \ldots, 380\). For any match \(m\), \(p_{m, H} + p_{m, D} + p_{m, A} = 1\),
but calculation of \(1/d_{m, H} + 1/d_{m, D} + 1/d_{m, A}\) for a given match \(m\) typically yields a value in the
region of 1.045 to 1.06, which highlights the use of over-round, \(o_{m, H}\), \(o_{m, D}\), and \(o_{m, A}\) in
Bet365's published odds. Values of \(1/d_{m, H}\), \(1/d_{m, D}\), and \(1/d_{m, A}\) and their sum are displayed in
Figure \ref{odds} via a stacked bar chart.
\begin{figure}
\begin{center}
\includegraphics[width = 14cm]{odds}
\caption{A stacked bar chart showing the value of \(1/d_{m, H} + 1/d_{m, D} + 1/d_{m, A}\) from the Bet365 1X2 betting
market for all 380 matches (\(m\)) in the 2011/2012 season. \protect\blueBox\ \(1/d_{m, H}\), \protect\redBox\ \(1/d_{m,
D}\), \protect\greenBox\ \(1/d_{m, A}\)}
\label{odds}
\end{center}
\end{figure}
In Chapter \ref{ch:An_adaptive_behaviour_model_for_association_football_using_rankings_as_prior_information} Section
\ref{sec:Use_of_models_to_inform_betting_strategies} we make the assumption that \(o_{m, H} = o_{m, D} = o_{m, A}\) in
order to estimate the bookmaker's assumed probabilities \(p_{m, H}\), \(p_{m, D}\), and \(p_{m, A}\).

\section{Aim and structure of the thesis}

This thesis aims to achieve the following points in the noted chapter:
\begin{enumerate}
\item \textit{The creation of models which are widely applicable, have improved predictive power, and are more parsimonious than
models currently in literature}. As mentioned throughout this introductory chapter, bookmakers need to offer hundreds of
markets for association football matches both before the match and in-play. We thus seek to create a model which is
applicable to a wide selection of markets, that is, a model of the goal times in a similar vein to
\cite{DixonRobinson1998}. We consider model parsimony important, and seek to create a more parsimonious, but still rich,
model. A model of this nature is introduced in Chapter
\ref{ch:An_adaptive_behaviour_model_for_association_football_using_rankings_as_prior_information}

\item \textit{Inference for models in a Bayesian framework}. The use of Bayesian methods is relatively rare in the
literature concerning association football models. We propose reasons for why the Bayesian framework is preferable for
this particular application in Chapter \ref{ch:Bayesian_computational_methods}, and then throughout the remainder of the
thesis, use Bayesian methods for all model inference

\item \textit{Methods of inference which are quick to compute}. We also view speed of computation as important, since
for in-play markets, updated predictions are needed continuously throughout time, in particular after goals. We thus
explore `particle filtering' methods in Chapter
\ref{ch:Fast_updating_of_dynamic_and_static_parameters_using_particle_filters} which also naturally present an
opportunity to add a dynamic element to the model, in a similar vein to \cite{Owen2011}

\item \textit{The creation of models which can capture behavioural aspects of teams}. Finally, we add another layer to
the model proposed in Chapter
\ref{ch:An_adaptive_behaviour_model_for_association_football_using_rankings_as_prior_information}, which represents how
teams change their behaviour in relation to their league situation. This research is able to investigate if a notion of
`value' can be placed on the different \gls{EPL} positions, and if there is evidence that concurrent matches are not
necessarily independent. This work is presented in Chapter \ref{ch:A_Utility_Based_Model}
\end{enumerate}
We also introduce the Bayesian methods used throughout the thesis in Chapter \ref{ch:Bayesian_computational_methods},
and present a conclusion to the thesis in Chapter \ref{ch:Conclusion}.

% The reasoning behind these aims, and the sections of the thesis which address them are as follows: 

% The use of Bayesian methods is relatively rare in the literature concerning association football models. We propose
% reasons of why the Bayesian framework is preferable for this particular application in Chapter
% \ref{ch:Bayesian_computational_methods}, and then throughout the remainder of the thesis, use Bayesian methods for all
% model inference.
% 
% As mentioned throughout this introductory chapter, bookmakers need to offer hundreds of markets for association football
% matches both before the match, and in-play. We thus seek to create a model which is applicable to a wide selection of
% markets, that is, a model of the goal times in a similar vein to \cite{DixonRobinson1998}. We consider model parsimony
% important, and seek to create a more parsimonious, but still rich, model. A model of this nature is introduced in
% Chapter \ref{ch:An_adaptive_behaviour_model_for_association_football_using_rankings_as_prior_information}.
% 
% We also treat speed of computation importantly, since for in-play markets, updated predictions are needed continuously
% throughout time, in particular after goals. We thus explore `particle filtering' methods in Chapter
% \ref{ch:Fast_updating_of_dynamic_and_static_parameters_using_particle_filters} which also naturally present an
% opportunity to add a dynamic element to the model, in a similar vein to \cite{Owen2011}.
% 
% Finally, we add another layer to the model proposed in Chapter
% \ref{ch:An_adaptive_behaviour_model_for_association_football_using_rankings_as_prior_information}, which represents how
% teams change their behaviour in relation to their league situation. This novel research is able to investigate if a
% value can be placed on the different \gls{EPL} positions and if there is evidence that concurrent matches are not
% necessarily independent. This work is presented in Chapter \ref{ch:A_Utility_Based_Model}.
% 
% To end the thesis, a conclusion is presented in Chapter \ref{ch:Conclusion}.
