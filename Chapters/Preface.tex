
\singlespacing
\chapter*{Abstract}
\onehalfspacing

This thesis presents advances in modelling and inference for match outcomes in the association football English Premier
League. We firstly extend earlier models by introducing a behavioural aspect which can be used to investigate how teams
react to the state of play in a match. We show that the model, in its simplest form, outperforms existing models and is
able to select a portfolio of profitable bets against a bookmaker. Secondly, we introduce a dynamic component to the
model by allowing team ability parameters to vary stochastically in time. We employ particle filtering methods to cope
with a mixture of static and dynamic parameters and find that the updating of posterior distributions is particularly
fast, a necessary attribute should we wish to update parameter estimates while matches are in-play. Furthermore, it is
shown that the methods are able to recover model parameters based on simulated league data. Finally, we propose an
extension to the model so that we are able to investigate how a team modifies its behaviour based on their league
situation. We consider league positions that are closely attainable and suggest that since teams modify their behaviour
based on their current league position, outcomes of different matches are not necessarily independent.
 
% This thesis represents advances in modelling and inference for match outcomes in the association football English
% Premier League. We firstly introduce an inhomogeneous Poisson-process model for the scoring times of competing home and
% away teams in association football matches which simplifies earlier models by representing a team's ability as a finite
% resource to be partitioned between attack and defence. The model represents behavioural aspects; in particular, we use
% the model to investigate whether and how teams modify their behaviour in response to the state of play in a match. We
% adopt a Bayesian approach to parameter estimation and, through the use of a novel prior, we take account of both
% historic information on team performance and current form in inferences of parameters and prediction of team
% performance. Using data from the English Premier League, we demonstrate that the model provides realistic predictions of
% the probabilities associated with match outcomes and can potentially highlight instances where bookmaker's odds may be
% underpriced.
% 
% Secondly, a class of techniques known as the particle filter is considered which provide Bayesian methods for updating
% our posterior belief around a dynamic system of parameters as new data becomes available. The dynamic system consists of
% parameters representing the resource/strength of each team in the English Premier League and thus readily allows for the
% teams' abilities to vary through time. In addition, there are static model parameters present which are also updated via
% particle filtering methods as the new data is observed. Special algorithms are necessary to allow for the combination of
% dynamic and static model parameters. We show that particle filtering methods allow for rapid updates of parameter
% posterior belief and can accurately track an unobserved dynamic system.
% 
% Finally, we extend the inhomogeneous Poisson-process model for the scoring times of competing home and away teams in
% association football matches so that each team's rate of scoring also depends on their current league situation. The
% model allows the interpretation of a utility value on the different league positions, where teams play offensively to
% score goals in order to reach positions of higher utility, or defensively in order to prevent conceding and moving to
% positions of lower utility. The framework suggests that concurrent matches are not independent, and news from one match
% can potentially affect the behaviour of teams in another. We employ various Bayesian techniques in order to perform
% inference and use data from five English Premier League seasons. We find plausible results regarding the utility of each
% league position, and furthermore, evidence that concurrent matches are not independent.

% \singlespacing
% \chapter*{Dedication}
% \onehalfspacing
%
% For Livingston Football Club. I will continue this work until I have saved up enough money to buy you some good players.

\singlespacing
\chapter*{Acknowledgements}
\onehalfspacing

I'd like to thank my supervisors, Professor Gavin Gibson and Doctor George Streftaris, from whom I have learned much
during my PhD.

On a more personal level, I'd like to thank my mum Margaret, my sister Emily, and my partner Lavinia, for pretending
to listen to me talk about a mixture of football and statistics for the past three years.



